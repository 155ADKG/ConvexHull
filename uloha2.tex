%%%%%%%%%%%%%%%%%%%%%%%%%%%%%%%%%%%%%%%%%
% Programming/Coding Assignment
% LaTeX Template
%
% This template has been downloaded from:
% http://www.latextemplates.com
%
% Original author:
% Ted Pavlic (http://www.tedpavlic.com)
%
% Note:
% The \lipsum[#] commands throughout this template generate dummy text
% to fill the template out. These commands should all be removed when 
% writing assignment content.
%
% This template uses a Perl script as an example snippet of code, most other
% languages are also usable. Configure them in the "CODE INCLUSION 
% CONFIGURATION" section.
%
%%%%%%%%%%%%%%%%%%%%%%%%%%%%%%%%%%%%%%%%%

%----------------------------------------------------------------------------------------
%	PACKAGES AND OTHER DOCUMENT CONFIGURATIONS
%----------------------------------------------------------------------------------------

\documentclass{article}

\usepackage[utf8]{inputenc}       %  sdělí, že další text je kódován v utf-8
\usepackage[czech]{babel}         %  zapne české dělení slov a české zvyklosti

\usepackage{fancyhdr} % Required for custom headers
\usepackage{lastpage} % Required to determine the last page for the footer
\usepackage{extramarks} % Required for headers and footers
\usepackage[usenames,dvipsnames]{color} % Required for custom colors
\usepackage{graphicx} % Required to insert images
\usepackage{listings} % Required for insertion of code
\usepackage{courier} % Required for the courier font
\usepackage{lipsum} % Used for inserting dummy 'Lorem ipsum' text into the template ... možný smazat ...

% Margins
\topmargin=-0.45in
\evensidemargin=0in
\oddsidemargin=0in
\textwidth=6.5in
\textheight=9.0in
\headsep=0.25in

\linespread{1.1} % Line spacing

% Set up the header and footer
\pagestyle{fancy}

%\lhead{\hmwkAuthorName} % Top left header
%\chead{\hmwkClass\ (\hmwkClassInstructor\ \hmwkClassTime): \hmwkTitle} % Top center head
\lhead{\hmwkClass: \hmwkTitle} % Top center head

\rhead{\firstxmark} % Top right header
\lfoot{\lastxmark} % Bottom left footer
\cfoot{} % Bottom center footer
\rfoot{Strana\ \thepage\ /\ \protect\pageref{LastPage}} % Bottom right footer
\renewcommand\headrulewidth{0.4pt} % Size of the header rule
\renewcommand\footrulewidth{0.4pt} % Size of the footer rule

\setlength\parindent{0pt} % Removes all indentation from paragraphs

%----------------------------------------------------------------------------------------
%	CODE INCLUSION CONFIGURATION
%----------------------------------------------------------------------------------------

\definecolor{MyDarkGreen}{rgb}{0.0,0.4,0.0} % This is the color used for comments
\lstloadlanguages{C++} % Load Perl syntax for listings, for a list of other languages supported see: ftp://ftp.tex.ac.uk/tex-archive/macros/latex/contrib/listings/listings.pdf
\lstset{language=C++, % Use Perl in this example
        frame=single, % Single frame around code
        basicstyle=\small\ttfamily, % Use small true type font
        keywordstyle=[1]\color{Blue}\bf, % Perl functions bold and blue
        keywordstyle=[2]\color{Purple}, % Perl function arguments purple
        keywordstyle=[3]\color{Blue}\underbar, % Custom functions underlined and blue
        identifierstyle=, % Nothing special about identifiers                                         
        commentstyle=\usefont{T1}{pcr}{m}{sl}\color{MyDarkGreen}\small, % Comments small dark green courier font
        stringstyle=\color{Purple}, % Strings are purple
        showstringspaces=false, % Don't put marks in string spaces
        tabsize=5, % 5 spaces per tab
        %
        % Put standard Perl functions not included in the default language here
        morekeywords={rand},
        %
        % Put Perl function parameters here
        morekeywords=[2]{on, off, interp},
        %
        % Put user defined functions here
        morekeywords=[3]{test},
       	%
        morecomment=[l][\color{Blue}]{...}, % Line continuation (...) like blue comment
        numbers=left, % Line numbers on left
        firstnumber=1, % Line numbers start with line 1
        numberstyle=\tiny\color{Blue}, % Line numbers are blue and small
        stepnumber=5 % Line numbers go in steps of 5
}

% Creates a new command to include a perl script, the first parameter is the filename of the script (without .pl), the second parameter is the caption
\newcommand{\perlscript}[2]{
\begin{itemize}
\item[]\lstinputlisting[caption=#2,label=#1]{#1.cpp}
\end{itemize}
}

%----------------------------------------------------------------------------------------
%	DOCUMENT STRUCTURE COMMANDS
%	Skip this unless you know what you're doing
%----------------------------------------------------------------------------------------

% Header and footer for when a page split occurs within a problem environment
\newcommand{\enterProblemHeader}[1]{
\nobreak\extramarks{#1}{#1 continued on next page\ldots}\nobreak
\nobreak\extramarks{#1 (continued)}{#1 continued on next page\ldots}\nobreak
}

% Header and footer for when a page split occurs between problem environments
\newcommand{\exitProblemHeader}[1]{
\nobreak\extramarks{#1 (continued)}{#1 continued on next page\ldots}\nobreak
\nobreak\extramarks{#1}{}\nobreak
}

\setcounter{secnumdepth}{0} % Removes default section numbers
\newcounter{homeworkProblemCounter} % Creates a counter to keep track of the number of problems

\newcommand{\homeworkProblemName}{}
{
\exitProblemHeader{\homeworkProblemName} % Header and footer after the environment
}

\newcommand{\problemAnswer}[1]{ % Defines the problem answer command with the content as the only argument
\noindent\framebox[\columnwidth][c]{\begin{minipage}{0.98\columnwidth}#1\end{minipage}} % Makes the box around the problem answer and puts the content inside
}

%----------------------------------------------------------------------------------------
%	NAME AND CLASS SECTION
%----------------------------------------------------------------------------------------

\newcommand{\hmwkTitle}{Konvexní obálka} % Assignment title
\newcommand{\hmwkDueDate}{17.\ prosince\ 2016} % Due date
\newcommand{\hmwkClass}{155ADKG - Algoritmy digitální kartografie a GIS} % Course/class
%\newcommand{\hmwkClassTime}{} % Class/lecture time
\newcommand{\hmwkClassInstructor}{Ing. Tomáš Bayer, Ph.D.} % Teacher/lecturer
\newcommand{\hmwkAuthorName}{Bc.\,Petr Bezděka, Bc.\,Šimon Gajzler, Bc.\,Lukáš Středa} % Your name

%----------------------------------------------------------------------------------------
%	TITLE PAGE
%----------------------------------------------------------------------------------------

\title{
\vspace{2in}
\textmd{\textbf{\hmwkClass:\ \hmwkTitle}}\\
\normalsize\vspace{0.1in}\small{Dne\ \hmwkDueDate}\\
\vspace{0.1in}\large{\textit{\hmwkClassInstructor}}
\vspace{3in}
}

\author{\textbf{\hmwkAuthorName}}
\date{} % Insert date here if you want it to appear below your name

%----------------------------------------------------------------------------------------

\begin{document}

\maketitle

%----------------------------------------------------------------------------------------
%	TABLE OF CONTENTS
%----------------------------------------------------------------------------------------

\setcounter{tocdepth}{1} % Uncomment this line if you don't want subsections listed in the ToC

\newpage
\tableofcontents
\newpage

%----------------------------------------------------------------------------------------
%	ZADÁNÍ
%----------------------------------------------------------------------------------------

% To have just one problem per page, simply put a \clearpage after each problem

%\section{Code C++}
%Text before \dots
%\begin{lstlisting}
%for (int i=0; i<iterations;i++)
%{
%do something
%}
%\end{lstlisting}
%Text after it \dots
%\end{document}

%Listing \ref{main} shows a C++ code.

\section{Zadání}
\emph{Vstup: množina} $P = \{p_1,\ldots, p_n\}, p_i = [x,y_i]$.\\

\emph{Výstup:} $\mathcal{H}(P)$.\\

Nad množinou $P$ implementujte následující algoritmy pro konstrukci $\mathcal{H}(P)$:

\begin{itemize}
\item Jarvis Scan.
\item Quick Hull.
\item Incremental Construction.\\
\end{itemize}

Vstupní množiny bodů včetně vygenerovaných konvexních obálek vhodně vizualizujte. Grafické rozhraní realizujte s~využitím frameworku QT. Dynamické datové struktury implementujte s využitím knihovny STL.\\
\\
Pro množiny $n\in <1000,1000000>$ vytvořte grafy ilustrující doby běhu algoritmů pro zvolená $n$. Měření proveďte pro různé typy vstupních množin (náhodná množina, rastr, clustrovaná data) opakovaně (10x) a různá $n$ (celkem 10) s~uvedením rozptylu. Naměřené údaje uspořádejte do přehledných tabulek.\\
\\
Zamyslete se nad problematikou možných singularit pro různé typy vstupních množin a možnými optimalizacemi. Zhodnoťte dosažené výsledky. Rozhodněte, která z~těchto metod je s~ohledem na~časovou složitost a typ vstupní množiny $P$ nejvhodnější.

%----------------------------------------------------------------------------------------
%	ROZBOR PROBLEMATTIKY
%----------------------------------------------------------------------------------------
\section{Rozbor problematiky}
Problematika určování konvexní obálky je v GIS systémech mnohdy používaná jako pomocná struktura pro řadu algoritmů. Má využití v prvním odhadu tvaru prostorového jevu, v analýze tvarů objektu, analýze shluků. V kartografii slouží pro detekci tvaru a natočení budov, což přináší užitek v případě generalizace.\\

Konvexní obálku chápeme jako minimální ohraničení určité množiny bodů. Žádný z těchto bodů nesmí být mimo obálku a pokud spojíme dva libovolné body z množiny úsečkou, tak se celá úsečka nachází uvnitř obálky.\\

V~přednáškách a cvičeních jsme se seznámili s algoritmy  pro výpočet konvexních obálek a v následující části se těmito algoritmy budeme zabývat.

\subsection{Quick Hull}
Princip metody Quick Hull je založen na konstrukci dvou obálek, horní (upper) a dolní (lower). Rozdělení na horní a dolní obálku se provádí pomocí spojnice dvou bodů s extrémními souřadnicemi $x$ a to $q_1 = min(x_i)$ a $q_3 = max(x_i)$. Horní je pak logicky nad touto spojnicí. V případě orientované úsečky z $q_1$ do $q_3$ jsou pak body v horní obálce vlevo od této spojnice. Následně nad každou nalezenou stranou obálky hledáme nejvzdálenější bod ležící vpravo od této spojnice. Takový bod se stává novým bodem obálky a stávající stranu rozdělí na dvě nové. Tento postup je aplikován do vyčerpání všech bodů a to v obou částech konvexní obálky (horní i~dolní) odděleně. Výsledná konvexní obálka je pak složena z obou částí.

\begin{figure}[ht]
\begin{center}
\includegraphics[width=0.6\columnwidth]{02quick} % Example image
\caption{Quick Hull algorithm}
\label{fig:wind}
\end{center}
\end{figure}

\subsection{Jarvis Scan}
Jarvis Scan je metoda popisovaná jako balení dárku. Předpokladem algoritmu je, že ve zkoumané množině nejsou tři body na jedné přímce. Algoritmus dále počítá, v případě již neseřazené množiny, s jejím předspracováním. A to setříděním množiny podle hodnoty y a x a nalezení pivotů $q$ a $s$, kde $q_y = min(y)$ a $s_x = min(x)$. 

\begin{figure}[h]
\begin{center}
\includegraphics[width=0.4\columnwidth]{02jarvis} % Example image
\caption{Jarvis Scan (Zdroj: https://pl.wikipedia.org/wiki/Algorytm\_Jarvisa/)}
\label{fig:ray}
\end{center}
\end{figure}

Algoritmus je založen na principu, kdy hledáme maximální vnitříní úhel mezi stranou tvořenou posledními dvěma body obálky a dalším nalezeným bodem, který není v obálce.

\subsection{Incremental Construction}
Incremental Construction je metoda založená na principu postupném přidávání dobů zkoumanné množiny do konvexní obálky. Tvar konvexní obálky je tak modifikován, přičemž mohou při přidávání bodů nastat dva případy. Prvním je možnost, že bod leží uvnitř konvexní obálky a její tvar neovlivní a druhá možnost, že bod leží vně obálky . Tím ovlivní její tvar a je nutné najít horní a dolní tečny, procházející přidávaným bodem, kolmé na konvexní obálku. Test polohy zkoumaného bodu je prováděn Half Edge testem.\\

Nalezení tečen se provádí testováním polohy zainteresovaných stran $e_j, e_{j+1}$ vzhledem k přidávanému bodu. Pro dolní tečnu platí, že přidávaný bod je vlevo od $e_j$ a v pravo od $e_{j+1}$. Pro horní tečnu patí, že bod je vpravo od $e_j$ a vlevo od $e_{j+1}$.

\subsection{Graham Scan}
Graham Scan je metoda, kde tři po sobě jdoucí body jsou součástí konvexní obálky v případě, kdy poslední bod leží vlevo od spojnice předchozích dvou. V případě kdy se nachází vpravo, dojde k odtranění prostředního bodu z konvexní obálky a následne se testují jeho dva předcházející body s jeho následníkem. Následují další testy, vždy se o jeden krok posuneme v množině bodů a znovu testujeme polohu posledního ze trojice bodů.\\

Podstatnou podnímkou pro fungování algoritmu je jeho předstpracování. To spočívá v nalezení pivota $q$, kde $q_y=min(y)$ a setřídění množiny bodů podle úhlu od osy $x$ ke zkoumanému bodu s vrcholem v bodě $q$.

\newpage
\section{Problematické situace}
\subsection{Kolineární situace u Jarvis Scanu}
Jak již byly popsáno v předešlé kapitole, algoritmus Jarvis Scanu hledá největší vnitří úhly mezi stranou tvořenou posledními dvěma body obálky a dalším bodem, který ještě není v obálce. Při tomto testu může nastat situace, kdy dva body mají stejný úhel (leží na jedné přímce). V takovém případě hledáme bod, který je nejdále od posledního bodu konvexní obálky.\\

V níže vloženém zdrojovém kódu testujeme podmínku, kdy vypočtený úhel je větší jak aktuální maximální úhel. V případě splnění podmínky je tento úhel uložen jako maximální. Při nesplnění podmínky dochází k testu, jestli si nejsou maximální a aktuální úhel rovny (kolineární situace). V takovém případě se spočte zvdálenost a posoudí se, která z hodnot je větší.\\
 
\begin{lstlisting}
 for (unsigned int i=0;i<points.size();i++)
        {
            double omg = getTwoVectorsOrientation(pj,pjj,pj,points[i]);
            if (omg > omg_max)
            {
                omg_max = omg;
                idx_max = i;
                d_max = sqrt((points[i].x()-pj.x())*(points[i].x()-pj.x()) +...
                + (points[i].y()-pj.y())*(points[i].y()-pj.y()));
            }
            else if (omg == omg_max)
            {
                double d = sqrt((points[i].x()-pj.x())*(points[i].x()-pj.x()) +...
                + (points[i].y()-pj.y())*(points[i].y()-pj.y()));
                if(d_max < d)
                {
                    omg_max = omg;
                    idx_max = i;
                    d_max = d;
                }
            }
        }
\end{lstlisting}
%----------------------------------------------------------------------------------------
%	UKÁZKA APLIKACE
%----------------------------------------------------------------------------------------
\newpage
\section{Ukázka vytvořené aplikace}
Aplikace obsahuje jedno tlačítko \emph{Generate}, kterým uživatel vygeneruje množinu bodů na základě jím zvolených parametrů. Tím je počet bodů, kam uživatel napíše kolik bodů se má vygenerovat, a typ generování, který si vybere z Comboboxu. Stiknutím tlačítka následně vykreslí aplikace množinu bodů do kreslící oblasti.\\

Obdobně funguje i vykreslení konvexní obálky. Uživatel má k dispozici Combobox, kterým zvolí metodu pro výpočet konvexní obálky a následně tento výpočet spustí tlačítkem \emph{Find Convex Hull}. Tím dojde k vykreslení konvexní obálky.
\begin{figure}[htbp]
\begin{center}
\includegraphics[width=0.50\columnwidth]{ukazka/random} % Example image
 \caption{Příklad vykreslené konvexní obálky nad body random}
\end{center}
\end{figure}

\begin{figure}[htbp]
\begin{center}
\includegraphics[width=0.50\columnwidth]{ukazka/grid} %Example image
 \caption{Příklad vykreslené konvexní obálky nad body v gridu}
\end{center}
\end{figure}

\begin{figure}[htbp]
\begin{center}
\includegraphics[width=0.50\columnwidth]{ukazka/cluster} %Example image
 \caption{Příklad vykreslené konvexní obálky nad body cluster}
\end{center}
\end{figure}
%---------------------------------------------- jarvis ----------------------------------------------
\section{Časová náročnost}
Následující tabulky zaznamenávají časovou náročnost jednotlivých algoritmů pro různé typy generovaných množin bodů v různém množství. Testy jsme prováděli na intervalu  $ <1000,1000000>$ pro typy generovaných množin bodů \emph{Random, Grid} a \emph{Cluster}. Každý test jsme ze statistických důvodů opakovali vždy 10x.\\

Nakonec byly spočteny průměrné hodnoty z deseti měření a vyhotoveny grafy znázorňující průběh časové náročnosti.
\begin{center}
	\begin{table}[]
	\centering
	\caption{Jarvis Scan - Random}
	\label{my-label}
	\begin{tabular}{|c|c|c|c|c|c|c|c|c|c|c|}
		\hline 
 	Počet bodů&  1000& 2500 & 5000 & 10 000 & 25 000 & 50 000  & 100 000  & 250 000  & 500 000 & 1 000 000  \\ \hline \hline
 		1. čas [ms]&  5&  13&  17&  31& 69 &    131&  253&  495&  909&  2041\\ \hline
 		2. čas [ms]&  5&    9&  22&  36&  68&    125&  208&  508&  907&  1757\\ \hline
 		3. čas [ms]&  6&  12&  16&  30&  65&    126&  209&  519&  939&  2032\\ \hline
 		4. čas [ms]&  6&  11&  16&  33&  61&    126&  216&  495&  899&  1751\\ \hline
 		5. čas [ms]&  5&  10&  19&  28&  71&   115 &  200&  490&  864&  1759\\ \hline
 		6. čas [ms]&  3&  12&  24&  34&  69&    124&  237&  550&  946&  1950\\ \hline
 		7. čas [ms]&  7&  13&  18&  28&  59&      97&  305&  536&  993&  1841\\ \hline
 		8. čas [ms]&  5&  11&  19&  34&  59&    105&  227&  514&  945&  1878\\ \hline
 		9. čas [ms]&  6&  11&  17&  26&  61&    130&  245&  487&  1049&  1858\\ \hline
		10. čas [ms]& 4 & 13 &  16&  28&  62&  113&  229&  579&  856&  1870\\ \hline \hline
 		Průměr [ms]& 5& 12 &18  &31  &54  &119  &233  &517  &931  & 1874 \\ \hline
	\end{tabular}
	%\end{table}

	%\begin{table}[ht]
	\centering
	\caption{Jarvis Scan - Grid}
	\label{my-label}
	\begin{tabular}{|c|c|c|c|c|c|c|c|c|c|c|}
		\hline 
 	Počet bodů&  1000& 2500 & 5000 & 10 000 & 25 000 & 50 000  & 100 000  & 250 000  & 500 000 & 1 000 000  \\ \hline \hline
 		1. čas [ms]& 2 & 11& 14& 18 &  40&  81&  155&  426&  856& 1826 \\ \hline
 		2. čas [ms]& 3&   5&  12&  22&  37&  80&  164&  425&  855& 1827\\ \hline
 		3. čas [ms]&  2& 7 &  13&  19&  36&  80&  157&  424&  909& 1828 \\ \hline
 		4. čas [ms]&  2&  5&  13&  18&  37&  80&  156&  425&  857& 1827 \\ \hline
 		5. čas [ms]&  2&  6&  14&  22&  36&  87&  157&  425&  915& 1826 \\ \hline
 		6. čas [ms]&  2&  6&  11&  17&  36&  84&  156&  427&  856& 1829 \\ \hline
 		7. čas [ms]&  3&  7&  9&    20&  37&  80&  156&  440&  856& 1829 \\ \hline
 		8. čas [ms]&  2&  6& 13&   19&  38&  80&  156&  425&  859& 1828 \\ \hline
 		9. čas [ms]&  2&  7&  9&    17&  36&  80&  156&  427&  859& 1827\\ \hline
		10. čas [ms]&  2& 3& 9&    19&  36&  81&  155&  427&  858& 1824 \\ \hline \hline
 		Průměr [ms]&  2&  6&  12&  19&  37&  81&  157&  427&  868&  1827\\ \hline
	\end{tabular}
%	\end{table}
%
%	\begin{table}[ht]
	\centering
	\caption{Jarvis Scan - Cluster}
	\label{my-label}
	\begin{tabular}{|c|c|c|c|c|c|c|c|c|c|c|}
		\hline 
 	Počet bodů&  1000& 2500 & 5000 & 10 000 & 25 000 & 50 000  & 100 000  & 250 000  & 500 000 & 1 000 000  \\ \hline \hline
 		1. čas [ms]&  5&  11&  19&  35&  91&  172&  306&  763&  1350&  2775\\ \hline
 		2. čas [ms]&  2&  11&  17&  36&  86&  151&  326&  804&  1548&  2488\\ \hline
 		3. čas [ms]&  5&  13&  16&  35&  89&  190&  329&  792&  1423&  2950\\ \hline
 		4. čas [ms]&  4&  13&  23&  32&  82&  184&  361&  753&  1357&  2848\\ \hline
 		5. čas [ms]&  5&  12&  17&  37&  74&  162&  337&  678&  1514&  2482\\ \hline
 		6. čas [ms]&  5&  14&  15&  26&  88&  186&  319&  808&  1500&  2701\\ \hline
 		7. čas [ms]&  5&  12&  18&  31&  84&  166&  310&  860&  1743&  2197\\ \hline
 		8. čas [ms]&  4&  11&  18&  34&  91&  166&  280&  832&  1504&  3066\\ \hline
 		9. čas [ms]&  6&   7 &  19&  34&  87&  194&  328&  825&  1592&  2849\\ \hline
		10. čas [ms]& 4& 10& 24&  39&  85&  180&  320&  765&  1698&  2903\\ \hline \hline
 		Průměr [ms]& 5 &11  & 19 & 34 & 86 &175  & 322 & 788 & 1523 & 2726 \\ \hline
	\end{tabular}
	\end{table}
\end{center}
%---------------------------------------------- quick hull ----------------------------------------------
\begin{center}
	\begin{table}[]
	\centering
	\caption{Quick Hull - Random}
	\label{my-label}
	\begin{tabular}{|c|c|c|c|c|c|c|c|c|c|c|}
		\hline 
 	Počet bodů&  1000& 2500 & 5000 & 10 000 & 25 000 & 50 000  & 100 000  & 250 000  & 500 000 & 1 000 000  \\ \hline \hline
 		1. čas [ms]&  3&  9&  10&   22&  28&  61&  118&  304&  562&  1093\\ \hline
 		2. čas [ms]&  3&10&  11&  16&  38&  72&  129&  302&  558&  1142\\ \hline
 		3. čas [ms]&  3&10&  13&  20&  32&  75&  143&  307&  588&  1020\\ \hline
 		4. čas [ms]&  3&  8&  11&  19&  37&  67&  146&  303&  590&  1086\\ \hline
 		5. čas [ms]&  3&  8&  13&  21&  37&  72&  148&  275&  619&  1089\\ \hline
 		6. čas [ms]&  3&  7&  14&  22&  38&  63&  144&  303&  618&  1207\\ \hline
 		7. čas [ms]&  2&  7&  13&  21&  33&  66&  112&  301&  559&  1087\\ \hline
 		8. čas [ms]&  3&  9&  10&  23&  40&  52&  130&  302&  556&  1025\\ \hline
 		9. čas [ms]&  2&  7&  16&  20&  36&  72&  134&  286&  558&  1023\\ \hline
		10. čas [ms]&4& 7&  11&  14&  34&  57&  150&  272&  561&  1096\\ \hline \hline
 		Průměr [ms]& 3 &8  &12 &20  &35  &66  &135  &296  &  577& 1087  \\ \hline
	\end{tabular}
%	\end{table}
%
%	\begin{table}[ht]
	\centering
	\caption{Quick Hull - Grid}
	\label{my-label}
	\begin{tabular}{|c|c|c|c|c|c|c|c|c|c|c|}
		\hline 
 	Počet bodů&  1000& 2500 & 5000 & 10 000 & 25 000 & 50 000  & 100 000  & 250 000  & 500 000 & 1 000 000  \\ \hline \hline
 		1. čas [ms]&  1&  2&  7&  10&  17&  43&  74&  191&  391&  848\\ \hline
 		2. čas [ms]&  1&  3&  4&  12&  17&  42&  73&  189&  391&  847\\ \hline
 		3. čas [ms]&  1&  4&  8&    8&  17&  42&  74&  190&  392&  851\\ \hline
 		4. čas [ms]&  1&  4&  8&  10&  18&  43&  74&  190&  392&  849\\ \hline
 		5. čas [ms]&  1&  4&  8&  13&  17&  43&  74&  191&  391&  849\\ \hline
 		6. čas [ms]&  2&  3&  7&  12&   18& 43&  74&  190&  392&  850\\ \hline
 		7. čas [ms]&  1&  3&  7&    8&   18& 43&  79&  190&  391&  846\\ \hline
 		8. čas [ms]&  1&  4&  8&    7&   22& 42&  74&  189&  391&  848\\ \hline
 		9. čas [ms]&  1&  4&  5&  10&   18& 42&  74&  189&  391&  847\\ \hline
		10. čas [ms]&1&  2&  8&  12&   17& 43&  74&  190&  392& 846\\ \hline \hline
 		Průměr [ms]&  1& 3 &7& 10 & 18 & 43 & 74 &  190&  391& 848 \\ \hline
	\end{tabular}
%	\end{table}
%
%	\begin{table}[ht]
	\centering
	\caption{Quick Hull - Cluster}
	\label{my-label}
	\begin{tabular}{|c|c|c|c|c|c|c|c|c|c|c|}
		\hline 
 	Počet bodů&  1000& 2500 & 5000 & 10 000 & 25 000 & 50 000  & 100 000  & 250 000  & 500 000 & 1 000 000  \\ \hline \hline
 		1. čas [ms]&  3&  7&  17&  18&  58&   94&  173&  383&  845&  1544\\ \hline
 		2. čas [ms]&  3&  6&  16&  21&  51&   92&  174&  400&  815&  1784\\ \hline
 		3. čas [ms]&  3&  6&  13&  17&  49&   94&  167&  354&  752&  1653\\ \hline
 		4. čas [ms]&  3&  9&    8&  19&  50& 118&  167&  462&  787&  1298\\ \hline
 		5. čas [ms]&  3&  7&  10&  18&  52&   94&  163&  449&  816&  1594\\ \hline
 		6. čas [ms]&  4&  9&  16&  24&  43&   81&  172&  384&  945&  1419\\ \hline
 		7. čas [ms]&  3&  9&  16&  21&  40&   91&  167&  383&  728&  1544\\ \hline
 		8. čas [ms]&  3&  8&  15&  19&  52&   92&  205&  414&  811&  1657\\ \hline
 		9. čas [ms]&  3&  8&  12&  21&  47&   84&  198&  465&  843&  1371\\ \hline
		10. čas [ms]&3&  8&  11&  26&  50&   89&  216&  417&  723&  1585\\ \hline \hline
 		Průměr [ms]&  3&  8&  13& 20 & 49 & 93 & 180 &411  &806  &1545  \\ \hline
	\end{tabular}
	\end{table}
\end{center}

\begin{figure}[htbp]
\begin{center}
\includegraphics[width=0.75\columnwidth]{jarvis} % Example image
 \caption{Časová náročnost Jarvis Scanu}
\end{center}
%\end{figure}
%
%\begin{figure}[h]
\begin{center}
\includegraphics[width=0.75\columnwidth]{qhull} % Example image
 \caption{Časová náročnost Quick Hull}
\end{center}
\end{figure}
%---------------------------------------------- quick hull ----------------------------------------------



%----------------------------------------------------------------------------------------
%	DOKUMENTACE
%----------------------------------------------------------------------------------------
\newpage
\section{Dokumentace \label{sec:alg}}
\subsection{Třída Algorithms}
Třída \emph{Algorithms} obsahuje funkce:
\begin{itemize}
\item jarvis
\item qhull
\item incr
\item graham
\item getPointLinePosition
\item getTwoVectorsOrientation
\item getPointLineDistance
\end{itemize}

\subsubsection{jarvis}
Do této funkce vstupuje jedna proměnná typu  \emph{std::vector$<$QPoint$>$}. Reprezentuje zkoumanou množinu bodů.\\
Výstupem funkce je opět množina bodů uspořádaná do \emph{std::vector$<$QPoint$>$}. Množina bodů obsahuje právě ty body, které tvoří konvexní obálku. Tato funkce ja následně volána ve třídě \emph{Draw} (str. \pageref{sec:draw}).
% opravit odkaz v textu

\subsubsection{qhull}
Do této funkce vstupuje jedna proměnná typu  \emph{std::vector$<$QPoint$>$}. Reprezentuje zkoumanou množinu bodů.\\
Výstupem funkce je množina bodů uspořádaná do \emph{std::vector$<$QPoint$>$}, reprezentující konvexní obálku. Tento vektor je následně volán ze třídy \emph{Draw}.

\subsubsection{incr}
Do této funkce vstupuje jedna proměnná typu  \emph{std::vector$<$QPoint$>$}. Reprezentuje zkoumanou množinu bodů.\\
Výstupem funkce je množina bodů uspořádaná do \emph{std::vector$<$QPoint$>$}, reprezentující konvexní obálku. Tento vektor je následně volán ze třídy \emph{Draw}.

\subsubsection{graham}
Do této funkce vstupuje jedna proměnná typu  \emph{std::vector$<$QPoint$>$}. Reprezentuje zkoumanou množinu bodů.\\
Výstupem funkce je množina bodů uspořádaná do \emph{std::vector$<$QPoint$>$}, reprezentující konvexní obálku. Tento vektor je následně volán ze třídy \emph{Draw}.

\subsubsection{getPointLinePosition}
Do této funkce vstupují tři proměnné typu \emph{QPoint}.\\
Výstupem funkce je hodnota typu integer, která reprezentuje polohu zkoumaného bodu (první proměnná) vůči vektoru daného druhou a třetí proměnnou. Výstup nabývá hodnoty 1 pokud je zkoumaný bod v levé polorovině od vektoru, 0 pokud je v pravé polorovině od vektoru a -1 pokud na přímce procházející vektorem.

\subsubsection{getTwoVectorsOrientation}
Do této funkce vstupují čtyři proměnné typu \emph{QPoint}.\\
Výstupem funkce je hodnota typu double, která reprezentuje úhel ve stupních mezi vektory, které jsou tvořeny první a druhou proměnnou a třetí a čtvrtou proměnnou.

\subsubsection{getPointLineDistance}
Do této funkce vstupují tři proměnné typu \emph{QPoint}.\\
Výstupem funkce je hodnota typu double, která reprezentuje vzdálenost zkoumaného bodu (první proměnná) vůči vektoru daného druhou a třetí proměnnou.

\subsection{Třída Draw \label{sec:draw}}
Třída Draw obsahuje privátní proměnné:
\begin{itemize}
\item \emph{std::vector$<$QPoint$>$} points;
\item \emph{std::vector$<$QPoint$>$} convexHull;
\item \emph{typeGen }typeGenerate;
\item \emph{typeAlg }typeAlgorithm;
\item \emph{unsigned int }countPoints;
\item \emph{bool }pushGen;
\item \emph{bool} pushAlg;
\item \emph{int }time\_alg;
\end{itemize}

Třída Draw obsahuje dále funkce:
\begin{itemize}
\item paintEvent
\item generatePoint
\item generatePoints
\item setTypeGenerate
\item setTypeAlgorithm
\item setCountPoints
\item setPushGen
\item setPushAlg
\item getTiming
\end{itemize}

\subsubsection{paintEvent}
Funkce je volána funkcí po kliknutí na tlačítko \emph{Generate} nebo \emph{Find Convex Hull}. Pomocí privátních proměnných \emph{pushAlg, pushGen} se zavolá funkce \emph{generatePoints}, která naplní proměnnou \emph{points} a vykreslí vygenerované body nebo se zavolá jeden z algoritmů, který naplní privátní proměnnou \emph{convexHull} a vykreslí ji včetně již vygenerovaných bodů.

\subsubsection{generatePoint}
Funkce generující jeden náhodný bod typu\emph{ QPoint}.

\subsection{generatePoints}
Funkce generuje body dle zvolené metody generování. Pokud se generuje typ cluster, vždy se vygeneruje prvotní bod (pivot), kolem kterého se generuje náhodný počet bodů od jedné do sta, poté se vygeneruje další pivot. při generování typu grid se přes dva for cykly (x a y) rovnoměrně rozdělí generovaná plocha pomocí bodů na čtvercovou síť bodů. Při typu random se pouze zavolá funkce \emph{generatePoint} ve for cyklu tolikrát, kolik nabývá hodnota proměnné \emph{countPoints}, tedy počet generovaných bodů.
\subsubsection{setTypeGenerate}
Metoda předává proměnnou z GUI do třídy \emph{Draw}.

\subsection{setTypeAlgorithm}
Metoda předává proměnnou z GUI do třídy \emph{Draw}.

\subsubsection{setCountPoints}
Metoda předává proměnnou z GUI do třídy \emph{Draw}.

\subsubsection{setPushGen}
Metoda předává proměnnou z GUI do třídy \emph{Draw}.

\subsubsection{setPushAlg}
Metoda předává proměnnou z GUI do třídy \emph{Draw}.

\subsubsection{getTiming}
Metoda vrací dobu výpočtu \ldots

\section{Třída Widget}
Třída \emph{Winget} obsahuje funkce:
\begin{itemize}
\item on\_pushGen\_clicked
\item on\_comboTypeGen\_currentIndexChanged
\item on\_pushAlg\_clicked
\item on\_comboTypeAlg\_currentIndexChanged
\end{itemize}

\subsubsection{on\_pushGen\_clicked}
Funkce se spouští kliknutím uživatele na tlačítko \emph{Generate} a příkazem \emph{repaint()} spustí funkci \emph{paintEvent} ve třídě \emph{Draw}.

\subsubsection{on\_comboTypeGen\_urrentIndexChanged}
Funkce se spouští změnou \emph{QComboBoxu} a předá informaci o uživatelem zvoleném typu generování bodů do třídy \emph{Draw}.

\subsubsection{on\_pushAlg\_clicked}
Funkce se spouští kliknutím uživatele na tlačítko \emph{Find Convex Hull} a příkazem \emph{repaint()} spustí funkci \emph{paintEvent} ve třídě \emph{Draw}.

\section{Třída sortByXAsc}
Třída \emph{sortByXAsc} obsahuje pouze operátor, který slouží funkcím z třídy \emph{Algorithms} k vzestupnému setřídění bodů podle souřdnice x.

\section{Třída sortByYAsc}
Třída \emph{sortByYAsc} obsahuje pouze operátor, který slouží funkcím z třídy \emph{Algorithms} k vzestupnému setřídění bodů podle souřdnice y.

\section{Třída sortDouble}
Třída \emph{sortDouble} obsahuje pouze operátor, který slouží funkci \emph{graham} z třídy \emph{Algorithms} k vzestupnému setřídění úhlů.

\section{Třída RemoveDuplicatePoints}
Třída \emph{RemoveDuplicatePoints} obsahuje pouze operátor, který slouží funkci \emph{incr} z třídy \emph{Algorithms} k vymazání duplicitních bodů.
%----------------------------------------------------------------------------------------
%	ZÁVĚR
%----------------------------------------------------------------------------------------
\newpage
\section{Závěr}
Aplikace byla vytvořena dle zadání v QtCreator. Algoritmy Jarvis Scan a Quick Hull na hledání konvexní obálky fungují bezchybně včetně ošetření kolineárního případu u Jarvis Scanu. \\

Z výsledků v tabulkách a grafu můžeme vypozorovat, že algoritmus Quick Hull je na testované množině mnohdy až dvakrát časově méně náročný. Dále lze pozorova různé časové náročnosti vzhledem k typu generovaných množin bodů.\\

K zamyšlení se nabízí možnosti optimalizace a především využití vícevláknových výpočtů, což by mnohonásobně zkrátilo časové náročnosti na velkých množinách.



%----------------------------------------------------------------------------------------
%	LITERATURA
%----------------------------------------------------------------------------------------
%\newpage
%\section{Literatura}
%
%[1] BAYER T., Přednáška koncexní obálky adk4, \emph{online}, 22.\_12.\_ 2016, dostupné na: http:adk.com  - přednášky \\
%
%[2] P. Martínek - BP
%----------------------------------------------------------------------------------------
%	POZNÁMKY
%----------------------------------------------------------------------------------------
%- zadání
%- popis a rozbor problému\\
%- popis algoritmů formálním jazykem\\
%- problematické situace a jejich rozbor (tj. simplexy) + ošetření těchto situací v kódu\\
%- vstupní data, formát vstupních dat, popis\\
%- výstupní data, formát výstupních dat, popis\\
%- printscreen vytvořené aplikace\\
%- dokumentaci: popis tříd, datových položek a jednotlivých metod\\
%- závěr, možné či neřešené problémy, náměty na vylepšení\\
%- seznamy literatury\\

%----------------------------------------------------------------------------------------

\end{document}